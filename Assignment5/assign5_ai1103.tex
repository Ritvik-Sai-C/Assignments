\documentclass{beamer}

% Theme choice:
\usetheme{Madrid}
\usepackage{setspace}
\usepackage{amssymb}
\usepackage{amsmath}
\usepackage{amsthm}
\usepackage{mathrsfs}
\usepackage{mathtools}
\usepackage{longtable}
\usepackage{graphicx}

\usepackage{listings}
    \usepackage{color}                                            %%
    \usepackage{array}                                            %%
    \usepackage{longtable}                                        %%
    \usepackage{calc}                                             %%
    \usepackage{multirow}                                         %%
    \usepackage{hhline}                                           %%
    \usepackage{ifthen}                                           %%
    \usepackage{lscape}     
    \usepackage{amsmath}
\def\inputGnumericTable{}
\newcommand{\mydet}[1]{\ensuremath{\begin{vmatrix}#1\end{vmatrix}}}
\newcommand{\myvec}[1]{\ensuremath{\begin{pmatrix}#1\end{pmatrix}}}
\providecommand{\pr}[1]{\ensuremath{\Pr\left(#1\right)}}
\providecommand{\cbrak}[1]{\ensuremath{\left\{#1\right\}}}
\providecommand{\brak}[1]{\ensuremath{\left(#1\right)}}
\newcommand*{\permcomb}[4][0mu]{{{}^{#3}\mkern#1#2_{#4}}}
\newcommand*{\perm}[1][-3mu]{\permcomb[#1]{P}}
\newcommand*{\comb}[1][-1mu]{\permcomb[#1]{C}}

% Title page details: 
\title{Assignment 5} 
\author{Ritvik Sai C - CS21BTECH11054}
\date{\today}

\begin{document}

% Title page frame
\begin{frame}
    \titlepage 
\end{frame}

\section{Question}
\begin{frame}{Question}

  Find the probability $P_s $ that in a men's tennis tournament the final match will last five sets.\\
(a) Assume that the probability p that a player wins a set equals 0.5.\\ 
(b) Use bayesian statistic with uniform prior (see law of succession).\\ 

\end{frame}

% Blocks frame
\section{Solution}
\begin{frame}[allowframebreaks]{Solution}
 
\item (a) Suppose that the probability P(A) that player A wins a set equals p=1-q. He wins the match in five sets if he wins two of the first four sets and the fifth set. Hence, the probability $ P_5(A)$ that he wins in five equals $ 6p^3q^2$. Similarly, the probability $p_5(B)$ that player B wins in five equals $ 6p^2q^3$.Hence,\\ 
              $p_5= p_5(A) + p_5(B) =6p^3q^2 +6p^2q^3 = 6p^2q^2$\\
is the probability that the match last five sets. \\
If p=q=1/2, then  p_5=3/8\\
\end{frame}


\begin{frame}{Solution}
\begin{align}
    

(b) Suppose now that $P(A) = \widetilde {p}$ is an RV with density f(p).In this case,\\
 $\widetilde {p_5}=6\widetilde {p^2}(1-\widetilde {p^2})$
is an RV. We wish to find its best bayesian estimate . Using the MS criterion, we obtain \\
$\hat {p_5}= E(p_5) = \[ \int_0^1 6p^2(1-p^2)f(p) \ dp \]$ \\
If f(p)=1, then  $\hat {p_5}=1/5 $ \\

\end{align} 
\end{frame}
\end{document}
