\documentclass{beamer}

% Theme choice:
\usetheme{Madrid}
\usepackage{setspace}
\usepackage{amssymb}
\usepackage{amsmath}
\usepackage{amsthm}
\usepackage{mathrsfs}
\usepackage{mathtools}
\usepackage{longtable}
\usepackage{graphicx}

\usepackage{listings}
    \usepackage{color}                                            %%
    \usepackage{array}                                            %%
    \usepackage{longtable}                                        %%
    \usepackage{calc}                                             %%
    \usepackage{multirow}                                         %%
    \usepackage{hhline}                                           %%
    \usepackage{ifthen}                                           %%
    \usepackage{lscape}     
    \usepackage{amsmath}
\def\inputGnumericTable{}
\newcommand{\mydet}[1]{\ensuremath{\begin{vmatrix}#1\end{vmatrix}}}
\newcommand{\myvec}[1]{\ensuremath{\begin{pmatrix}#1\end{pmatrix}}}
\providecommand{\pr}[1]{\ensuremath{\Pr\left(#1\right)}}
\providecommand{\cbrak}[1]{\ensuremath{\left\{#1\right\}}}
\providecommand{\brak}[1]{\ensuremath{\left(#1\right)}}
\newcommand*{\permcomb}[4][0mu]{{{}^{#3}\mkern#1#2_{#4}}}
\newcommand*{\perm}[1][-3mu]{\permcomb[#1]{P}}
\newcommand*{\comb}[1][-1mu]{\permcomb[#1]{C}}

% Title page details: 
\title{Assignment 4} 
\author{Ritvik Sai C - CS21BTECH11054}
\date{\today}

\begin{document}

% Title page frame
\begin{frame}
    \titlepage 
\end{frame}

\section{Question}
\begin{frame}{Question}
\begin{itemize}
    \item The random variables $X_i$ are i.i.d with density $ce^{-cx} U(x)$. Show that, if $x = x_1+\dots+x_n$, then $f_x(x)$ is an Erlang density.  
\end{itemize}
\end{frame}

% Blocks frame
\section{Solution}
\begin{frame}[allowframebreaks]{Solution}
 Since, we know that,
\begin{align}
 \text{When } f(x) &= \gamma x^{b-1} e^{-cx} U(x) \text{ and } \gamma = \frac{c^{b+1}}{\Gamma (b+1)} \\
\text{It follows that } \phi (s) &= \gamma \int_{0}^{\infty} x^{b-1} e^{-(c-s)x} dx = \frac{\gamma \Gamma (b)}{\brak{c-s}^b} = \frac{c^b}{\brak{c-s}^b} 
\end{align}

Differentiating with respect to $s$ and setting $s = 0$, we obtain
\begin{align}
    \phi^{n} (0) &= \frac{b(b+1)\dots(b+n-1)}{c^n} = E\brak{x^n}
\end{align}
    
With $n = 1$ and $n = 2$, this yields 
    \begin{align}
        E\brak{x} &= \frac{b}{c} \\
        E \brak{x^2} &= \frac{b(b+1)}{c^2}\\
        \sigma^2 &= \frac{b}{c^2}
    \end{align}
    
    The exponential density is a special case obtained with $b = 1, C = \lambda$: 
    \begin{align}
        f(x) &= \lambda e^{-\lambda x} U(x) \\
        \phi (s) &= \frac{\lambda}{\lambda - s} 
    \end{align}
    where $E\brak{x} = \frac{1}{\lambda}$ and  $\lambda^2 = \frac{1}{\sigma^2}$ \\ 
Now, It is given that, 
\begin{align}
f_1(x) &= ce^{-cx} U(x) \\
\text{then } \phi_1(s) &= \frac{c}{c-s}\\
\phi(s) &= \phi_1(s) + \phi_2(s) + \dots + \phi_n(s) = \frac{c^n}{\brak{c-s}^n}\\
   \text{Hence, } f(x) &= \frac{c^n x^{m-1}}{\brak{n-1}!} e^{-cx} U(x)  
    \end{align}
\end{frame} 
\end{document}